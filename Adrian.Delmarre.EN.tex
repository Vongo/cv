\documentclass[fontsize=10pt]{tccv}
\usepackage[english]{babel}

\begin{document}

\part{Adrian Delmarre}

\section{Work experience}

\begin{eventlist}

\item{February 2015 -- Present}
     {Universit\ddot{a}t Passau, Passau, Germany}
     {Fraud Detection (Lab Internship)}
The purpose of the project is to evaluate and improve current techniques for detection of payment fraud. This involves high-volume real-time machine learning issues.
Skills : Binary classification, R
\item{May 2014 -- September 2014}
     {ANEO, Boulogne-Billancourt, France}
     {Social Network Analyse (R\&D Internship)}
By examining interpersonal communications (mainly email) as typed graphs using Social Network Analysis techniques, we got a better understanding of the structure of inner and outer-organizational communication. This helped us improving the way information and change could be brought.
Skills : Data Analysis, R, Social Network Analysis, Organizational Network Analysis
\item{June 2013 -- September 2013}
     {Systar, Lyon, France}
     {NoSQL Database Backoffice Development (R\&D Internship)}

Conception and implementation of a RedoLog.
In a NoSQL (column, key-value) bi-temporal database context, the internship objective was to log transactions in a custom binary format, so that we could guarantee platform integrity even after a crash, by replaying transactions in the same exact order.
Skills : Java EE, Concurrency programming, NoSQL, Scrum, IntelliJ Idea
\item{January 2012 -- March 2012}
     {P\hat{o} le Azur Provence, Grasse, France}
     {Product Development (Internship)}

I was asked to design and develop, from scratch, a Sharepoint Workflow management tool. This was made possible by the combination of a desktop user interface and a Windows Service. They were both run on the Sharepoint main server, which involved memory and performance management issues.
Skills : Project Management, Sharepoint, C\#, WinService, Visual Studio.
\item{May 2011 -- July 2011}
     {City Hall, Marseille, France}
     {J2EE Development (Internship)}

The Department of Public Contracts needed a monitoring and management tool.
I personally took part in conception meetings with the client, wrote the technical specifications, and eventually developed a few components.
Skills : Java J2EE, Hibernate, Struts, Tomcat, Eclipse, WinDesign.
\end{eventlist}
\personal
     {}
     {188 avenue Claude Farr\grave{e} re, 83000 Toulon (France)}
     {+33 674 155 147}
     {adelmarre@vongo.eu}
\section{Education}
\begin{list}
\item{2013--2015}
     {Computer Science Master Degree - Expected}
     {Universit\ddot{a} t Passau, Germany}
\item{2012--2015}
     {Computer Science Engineer Degree - Expected}
     {INSA Lyon, France}
\item{2010--2012}
     {Computer Science Technical Degree (BTS IG)}
     {LTP Charles P\acute{e} guy, Marseille, France}
\item{2008--2010}
     {Bachelor in Epistemology}
     {Universit\acute{e} d'Aix-en-Provence, France}
\item{2006--2008}
     {Classe pr\acute{e} paratoire, Literature and Humanities}
     {Dumont d'Urville, Toulon, France}
\item{2006}
     {Mathematics High School Diploma, Summa Cum Laude}
     {lalala}
\end{list}
\section{Main side-projects}
\begin{list}
\item{2014}
     {WittTheFact}
     {Using both my Computer Science and Epistemology skills, I have try to develop a
     FactBase engine based on Wittgenstein’s Tractatus Logico-Philosophicus.
     The final purpose of such an engine was to allow a faster processing for some specific data-mining analysis cases, while keeping a reliable model regarding ACID criteria.}
\item{2013}
     {Autonomee}
     {With a few classmates, we've built a little robot-car using the power of Arduino, Raspberry Pi and a couple of motors associated with a bunch of sensors. On top of that, we've also built a desktop (and a mobile) client able to control the car remotely (with the ability to automatically avoid obstacles), but also simulated a particle filter: an algorithm that makes it possible to determine the position and orientation of the car even though the sensors are noisy.}
\end{list}
\section{Communication skills}
\begin{factlist}
\item{French}
     {Native speaker}
\item{English}
     {Fluent}
\item{Spanish}
     {Oral: good}
\item{German}
     {Oral: fair}
\end{factlist}
\section{Skills}
\begin{factlist}
\item{Good level}
     {R, C, GNU-Linux}
\item{Intermediate}
     {\LaTeX, SQL, C++, Python, Java, C\#, git, UML, Data Mining Analysis}
\item{Basic knowledge}
     {Scala, HTML/CSS, Javascript}
\end{factlist}
\end{document}
